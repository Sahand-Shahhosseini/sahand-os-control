\documentclass[11pt,a4paper]{article}

% =======================
% Packages (conservative)
% =======================
\usepackage[T1]{fontenc}
\usepackage[utf8]{inputenc}
\usepackage{lmodern}
\usepackage{microtype}
\usepackage{geometry}
\geometry{top=2.4cm,bottom=2.4cm,left=2.4cm,right=2.4cm}
\usepackage{xcolor}
\usepackage{hyperref}
\hypersetup{
  colorlinks=true,
  linkcolor=black,
  urlcolor=black,
  citecolor=black,
  pdfauthor={Sahand Canon},
  pdftitle={DOC\_PROTO\_7\_PLUS\_14\_MAINSTREAM v0.3 Integrated RC1}
}
\usepackage{enumitem}
\usepackage{booktabs}
\usepackage{tabularx}
\usepackage{longtable}
\usepackage{array}
\usepackage{tcolorbox}
\tcbset{before skip=8pt, after skip=8pt, boxrule=0.6pt, arc=2pt}

% =======================
% Styling helpers
% =======================
\definecolor{GovRed}{RGB}{120,0,0}
\definecolor{ScaleBlue}{RGB}{0,70,150}
\definecolor{SoftGray}{RGB}{245,245,245}

\newtcolorbox{GovernanceBox}[1]{
  colback=white,
  colframe=GovRed,
  title=\textbf{Governance Authority: #1},
  coltitle=white,
  fonttitle=\bfseries,
  boxrule=0.9pt,
  left=6pt,right=6pt,top=6pt,bottom=6pt
}
\newtcolorbox{ScaleBox}[1]{
  colback=SoftGray,
  colframe=ScaleBlue,
  title=\textbf{Scale Clauses: #1},
  coltitle=white,
  fonttitle=\bfseries,
  boxrule=0.8pt,
  left=6pt,right=6pt,top=6pt,bottom=6pt
}
\newtcolorbox{TemplateBox}[1]{
  colback=white,
  colframe=black,
  title=\textbf{Template: #1},
  coltitle=white,
  fonttitle=\bfseries,
  boxrule=0.6pt,
  left=6pt,right=6pt,top=6pt,bottom=6pt
}

\setlist[itemize]{leftmargin=*, itemsep=3pt, topsep=4pt}
\setlist[enumerate]{leftmargin=*, itemsep=3pt, topsep=4pt}

% =======================
% Title
% =======================
\title{\textbf{\LARGE DOC\_PROTO\_7\_PLUS\_14\_MAINSTREAM}\\[2pt]
\large Integrated Specification (v0.3_Integrated_RC2)}
\author{Protocol Specification (Integrated Mainstream Standard)}
\date{2026-02-02}

\begin{document}
\maketitle
\tableofcontents
\newpage

% =============================================================================
\section{Scope, Design Intent, and Non-Goals}
\subsection{Purpose}
This document specifies \textbf{DOC\_PROTO\_7\_PLUS\_14\_MAINSTREAM} as a self-contained, reviewable standard that combines:
\begin{itemize}
  \item \textbf{Form Layer (DOC-7):} artifact governance, traceability, and execution discipline.
  \item \textbf{Content Layer (Math-7 + Domain-7):} scientific/mathematical content and domain-grounded operationalization.
\end{itemize}

\subsection{Mainstream constraint}
The protocol must be usable by external reviewers without knowledge of any private ecosystem-specific tooling. Project-specific audit engines are permitted as optional instantiations (Section~\ref{sec:audit_authority}), but the protocol itself remains vendor-agnostic.

\subsection{Non-goals}
\begin{itemize}
  \item This specification does \emph{not} prescribe a particular research field; Domain-7 is polymorphic.
  \item This specification does \emph{not} assume any proprietary agent; Import/selection logic may exist externally but is not required to understand compliance.
  \item This specification does \emph{not} replace discipline-specific regulations; it defines hooks (D7 gates) to enforce them.
\end{itemize}

% =============================================================================
\section{Architecture: Layers, Blocks, and Lifecycle}
\subsection{Three-layer composition}
A compliant project consists of 21 blocks:
\begin{enumerate}
  \item \textbf{DOC-7 (Form/Governance):} DOC01..DOC07.
  \item \textbf{Math-7 (Abstract Content):} M1..M7.
  \item \textbf{Domain-7 (Concrete Content):} D1..D7.
\end{enumerate}

\subsection{Artifact lifecycle states}
Every DOC artifact and every executable deliverable must declare a lifecycle state:
\begin{itemize}
  \item \textbf{DRAFT:} unstable, internal iteration.
  \item \textbf{REGISTERED:} design frozen before data collection/execution (Registered Reports pattern).
  \item \textbf{RELEASED:} reviewed, versioned, stable.
  \item \textbf{DEPRECATED:} retained for history; not valid for new work.
\end{itemize}

\subsection{The Form--Content separation invariant}
DOC blocks are \textbf{containers and governors}. They must not introduce new mathematics or new domain measurements; they only:
\begin{itemize}
  \item govern consistency, quality, and release gates,
  \item orchestrate mapping, traceability, and execution.
\end{itemize}

% =============================================================================
\section{Global Invariants (Fail-Closed)}
\label{sec:global_invariants}
A project claiming compliance \textbf{MUST} satisfy all invariants below. Failure is \textbf{fail-closed}: artifacts are invalid until corrected.

\begin{enumerate}
  \item \textbf{Strict Form/Content Separation:} DOC blocks do not create new math or domain claims.
  \item \textbf{Single Source of Truth for Symbols (DOC01):} any symbol/type mismatch across artifacts is a critical defect.
  \item \textbf{Mandatory Bridge (D3):} no domain claim is valid without D3 mapping; each row includes validation and failure conditions.
  \item \textbf{Noise-Aware Bridge (D3):} each mapping row includes a measurement/noise model and sensitivity/error propagation.
  \item \textbf{Conjecture Ledger (M7):} all unproven claims underpinning deliverables are registered with testability and lifecycle.
  \item \textbf{Zombie Conjecture Prevention (M7):} each conjecture has blocking status, owner, expiry/due date, and verdict state.
  \item \textbf{Algorithm--Implementation Separation:} M6 is theory-only; D6 is implementation-only; no code in M6, no new math in D6.
  \item \textbf{Cross-cutting Safety (D7 Gates):} privacy/ethics/security/traceability gates apply across blocks, not only at the end.
  \item \textbf{Reproducibility Baseline (All scales):} seeds, environment lock, and evidence manifest are mandatory even in MIN scale.
  \item \textbf{Explicit Objective (DOC06):} goals must be explicit, including trade-off rules under conflict.
\end{enumerate}

% =============================================================================
\section{Scale Model (MIN / STANDARD / FULL)}
\label{sec:scale_model}
Projects are categorized into:
\[
\text{ProjectScale} \in \{\textbf{MIN},\ \textbf{STANDARD},\ \textbf{FULL}\}.
\]
\begin{itemize}
  \item \textbf{MIN:} prototypes and solo research. Core correctness and reproducibility.
  \item \textbf{STANDARD:} publication-grade work (e.g., Q1-level). Strong justification and analysis discipline.
  \item \textbf{FULL:} safety-critical or production systems. Formal traceability and regulatory readiness.
\end{itemize}

\subsection{Math proof scaling}
\label{sec:math_proof_scaling}
\begin{table}[h]
\centering
\begin{tabularx}{\textwidth}{@{}lX@{}}
\toprule
\textbf{Scale} & \textbf{Math-7 proof requirement (for nontrivial, load-bearing claims)} \\
\midrule
MIN & Either (i) cite an established proof (with precise theorem ID), or (ii) register the claim in M7 as conjectural. No silent handwaving. \\
STANDARD & Provide a structured sketch of proof: key lemmas, dependencies, assumptions, and failure regimes. \\
FULL & Provide a full proof at a level sufficient for hostile review; formalization is encouraged where feasible. \\
\bottomrule
\end{tabularx}
\caption{Proof scaling rules for Math-7.}
\end{table}

% =============================================================================
\section{Canonical Mapping: Form (DOC) to Content (Math/Domain)}
\label{sec:mapping_table}
The relationship between DOC artifacts and content blocks is \textbf{not} one-to-one. The canonical mapping is:

\begin{table}[h]
\centering
\begin{tabularx}{\textwidth}{@{}lX X@{}}
\toprule
\textbf{DOC Artifact} & \textbf{Primary Content Blocks} & \textbf{Secondary / Optional Blocks} \\
\midrule
DOC01 (Foundations) & M1, M2, M3 & D3 (high-level bridge constraints) \\
DOC02 (Metatheory) & M5, M7 & DOC06 (value tensions), D7 (risk rationale) \\
DOC03 (Whitepaper) & D1, D2, D6 (high-level) & M4 (high-level), M7 (explicitly conjectural) \\
DOC04 (Proposal/Plan) & D2, D4, D7 & M7 (blocking dependencies), DOC07 (execution plan) \\
DOC05 (Technical Paper) & M1--M6, D3--D5 & M7, D7 (limitations, safety) \\
DOC06 (Value/Goal Map) & all blocks (as optimization constraints) & --- \\
DOC07 (Execution/OS) & D6, D4 (test harness), D7 (ops gates) & M6 (algorithm IDs), DOC01 (schema checks) \\
\bottomrule
\end{tabularx}
\caption{Canonical injection of content blocks into form artifacts.}
\end{table}

% =============================================================================
\section{Layer 1: DOC-7 (Form, Governance, and Enforcement)}
\subsection{DOC01: Foundations Specification (System Constitution)}
\begin{GovernanceBox}{System Constitution \& Symbol Law}
DOC01 is the \textbf{single source of truth} for:
\begin{itemize}
  \item symbol dictionary (types, units where applicable, domains/ranges),
  \item notation and style rules,
  \item consistency enforcement policy (what counts as a critical mismatch).
\end{itemize}
\textbf{Hard rule:} If $x$ is declared a vector in DOC01, it cannot be a scalar in code or in other documents. Any inconsistency is a \textbf{critical bug}.
\end{GovernanceBox}

\textbf{Minimum required sections:} Symbol Dictionary; Notation Standards; Type/Unit Rules; Consistency Policy; Version/Change Log.
\textbf{Machine-Readable Registry (Normative for FULL-Spec):} This release includes \texttt{protocol\_registry.sqlite} and its schema \texttt{protocol\_registry\_schema.sql}. The SQLite registry is the canonical machine-readable mirror of:
(i) the DOC01 Symbol Dictionary (\texttt{symbol\_dictionary}) and
(ii) the M7 Conjecture Ledger (\texttt{conjecture\_ledger}).
\textbf{Hard rule:} any divergence between DOC01/M7 text and the registry is a \textbf{Critical Bug} until resolved.

\begin{ScaleBox}{DOC01}
\begin{itemize}
  \item MIN: core symbol dictionary + style rules.
  \item STANDARD: explicit type/unit/range for load-bearing symbols; conflict resolution rules.
  \item FULL: machine-readable schema (for linters/code-gen) + automated consistency report.
\end{itemize}
\end{ScaleBox}

\subsection{DOC02: Metatheory \& Rationale (Internal Auditor)}
\begin{GovernanceBox}{Internal Auditor \& Rejected Alternatives}
DOC02 provides an adversarial justification:
\begin{itemize}
  \item why design choice X was selected,
  \item why alternatives were rejected,
  \item known weaknesses and expected failure regimes.
\end{itemize}
\textbf{Hard rule:} DOC02 must explicitly list \textbf{Rejected Alternatives} and \textbf{Known Weaknesses}. No ``story-only'' metatheory is acceptable.
\end{GovernanceBox}

\begin{ScaleBox}{DOC02}
\begin{itemize}
  \item MIN: concise rationale + at least 3 rejected alternatives + explicit limitations.
  \item STANDARD: structured defense against reviewer objections; linkage to M7 blocking items.
  \item FULL: hostile-audit-level critique, including operational risk analysis tied to D7 gates.
\end{itemize}
\end{ScaleBox}

\subsection{DOC03: Whitepaper (Narrative Gatekeeper)}
\begin{GovernanceBox}{Claim Discipline}
DOC03 is a public-facing narrative artifact.
\textbf{Hard rule:} every claim in DOC03 must link to either:
\begin{itemize}
  \item a proof element in Math-7 (M1--M6), or
  \item a measured/observed statement justified via Domain-7 (D3--D5), or
  \item a conjecture explicitly labeled and registered in M7.
\end{itemize}
\end{GovernanceBox}

\begin{ScaleBox}{DOC03}
\begin{itemize}
  \item MIN: concise; no unlinked claims.
  \item FULL: includes risk/limitations summary consistent with D7 and DOC02.
\end{itemize}
\end{ScaleBox}

\subsection{DOC04: Proposal \& Plan (PMO)}
\begin{GovernanceBox}{Project Management Office (PMO)}
DOC04 governs execution planning.
\textbf{Hard rule:} DOC04 must identify a \textbf{critical path} and list blockers coming from M7 (CRITICAL conjectures).
\end{GovernanceBox}

\textbf{Minimum required sections:} Work Breakdown Structure (WBS); milestones; critical path; resource constraints; risk register (linked to D7); acceptance tests.

\begin{ScaleBox}{DOC04}
\begin{itemize}
  \item MIN: minimal WBS + milestones + top 3 risks.
  \item STANDARD: critical path + dependency graph + risk register.
  \item FULL: budget/time constraints + formal verification/validation plan and traceability targets.
\end{itemize}
\end{ScaleBox}

\subsection{DOC05: Technical Paper (Report Engine \& QA)}
\begin{GovernanceBox}{Assembly Manager \& Quality Gate}
DOC05 is the report engine (IMRAD-aligned).
\textbf{Hard rules:}
\begin{itemize}
  \item include a \textbf{Content Mapping Table} mapping paper sections $\to$ blocks (M/D/DOC),
  \item include a \textbf{Reproducibility Statement} (seed, environment, artifact map),
  \item include a \textbf{Limitations} section referencing M7 and D7 gates.
\end{itemize}
\end{GovernanceBox}

\begin{ScaleBox}{DOC05}
\begin{itemize}
  \item MIN: short methods + minimal reproducibility statement.
  \item STANDARD: full IMRAD with statistical plan and uncertainty reporting.
  \item FULL: registered design (where applicable), full traceability, and safety/regulatory appendix.
\end{itemize}
\end{ScaleBox}

\subsection{DOC06: Value \& Goal Map (Objective Authority)}
\begin{GovernanceBox}{Optimization Objective \& Trade-off Law}
DOC06 defines the objective(s) and conflict resolution.
\textbf{Hard rule:} provide an explicit \textbf{Trade-off Matrix} stating what wins under conflict (e.g., safety over speed; accuracy over throughput).
\end{GovernanceBox}

\begin{ScaleBox}{DOC06}
\begin{itemize}
  \item MIN: qualitative weights (High/Medium/Low) + explicit priorities.
  \item STANDARD: numeric weights + documented trade-offs + KPI definitions.
  \item FULL: multi-objective formulation + acceptance thresholds tied to D7 gates and release criteria.
\end{itemize}
\end{ScaleBox}

\subsection{DOC07: Execution / OS Spec (Runtime Authority)}
\begin{GovernanceBox}{DevOps, Rebuild, and Release Authority}
DOC07 governs the bridge from design to execution.
\textbf{Hard rules:}
\begin{itemize}
  \item define official entry points (how to run),
  \item define artifact map (where everything lives),
  \item lock the environment (dependency versions) for reproducibility,
  \item define test harness (how D4 protocols run as tests),
  \item enforce D7 operational gates (privacy/security/monitoring).
\end{itemize}
\end{GovernanceBox}

\textbf{Minimum required sections:} Artifact Map; Entry Points; Environment Spec; Dependency Lock; Test Harness; Release/Versioning Strategy; Evidence Manifest plan.

\begin{ScaleBox}{DOC07}
\begin{itemize}
  \item MIN: artifact map + single entry point + minimal environment notes.
  \item STANDARD: pinned dependencies + automated tests + evidence manifest.
  \item FULL: CI/CD strategy + deployment spec + monitoring + incident response and decommission plan.
\end{itemize}
\end{ScaleBox}

% =============================================================================
\section{Layer 2: Math-7 (Abstract Content)}
The Math-7 layer is the \textbf{formal content engine}. It must align with DOC01 and respect proof scaling (Section~\ref{sec:math_proof_scaling}).

\subsection{M1: Mathematical Objects \& Spaces}
\textbf{Required:} object definitions, state spaces, topology/measure structure (as appropriate), typing rules aligned with DOC01.

\subsection{M2: Dynamics / Operators / Rules}
\textbf{Required:} evolution rules, operators, admissibility conditions, and invariants they preserve.

\subsection{M3: Theorems / Invariants / Proofs}
\textbf{Required:} statements + proof content scaled by Section~\ref{sec:math_proof_scaling}. Any unproven load-bearing claim must be moved to M7.

\subsection{M4: Models (Mathematical Representation)}
\textbf{Required:} model class, assumptions, identifiability/regularity assumptions where relevant, links to D3 for observables.

\subsection{M5: Links to Established Theory}
\textbf{Required:} explicit mapping to known theories; boundary of novelty; citations sufficient for review.

\subsection{M6: Algorithms (Theory Only)}
\begin{GovernanceBox}{Algorithm Theory Only (No Code)}
M6 specifies algorithms as \textbf{theory}:
\begin{itemize}
  \item each algorithm has an \textbf{ALG-ID},
  \item input/output contract, correctness argument, convergence conditions,
  \item complexity characterization, and failure regimes.
\end{itemize}
\textbf{Hard rule:} no code and no code-like pseudocode. Only \emph{algorithm schema} (natural-language steps + DOC01 notation).
\end{GovernanceBox}

\subsection{M7: Conjecture Ledger (Lifecycle Registry)}
\begin{GovernanceBox}{Anti-Zombie Conjecture Registry}
Each conjecture entry must include:
\begin{itemize}
  \item CJ-ID, statement, testability, \textbf{blocking status} (CRITICAL/IMPORTANT/OPTIONAL),
  \item owner, due/expiry date, current evidence, next action,
  \item verdict state (OPEN/VERIFIED/FALSIFIED/DEPRECATED).
\end{itemize}
\textbf{Hard rule:} any deliverable that depends on a CRITICAL OPEN conjecture cannot be RELEASED.
\textbf{Note:} this gate is machine-checkable via the SQLite registry.
\end{GovernanceBox}

% =============================================================================
\section{Layer 3: Domain-7 (Concrete Content)}
Domain-7 operationalizes the theory in a concrete field (medicine, software, physics, etc.). It is responsible for observability, data reality, and validation.

\subsection{D1: Domain Background}
\textbf{Required:} state of the art, standards/guidelines, known constraints, and what is measurable.

\subsection{D2: Problem Statement \& Use Cases}
\textbf{Required:} bounded, measurable problem definition; success criteria; use cases; failure cases.

\subsection{D3: Math-to-Domain Bridge (Noise-Aware)}
\begin{GovernanceBox}{Bridge with Noise Model and Sensitivity}
D3 is mandatory. Each mapping row must include:
\begin{itemize}
  \item units/scaling and valid ranges,
  \item observability tag (Observed vs Latent),
  \item measurement/noise model and uncertainty parameters,
  \item sensitivity/error propagation and error tolerance,
  \item validation metric and failure modes.
\end{itemize}
\end{GovernanceBox}

\subsection{D4: Experimental / Clinical / Simulation Protocols}
\begin{GovernanceBox}{Registered Design Discipline}
D4 defines data-generation/collection protocols.
\textbf{Hard rule:} even in MIN scale, stochastic protocols must declare seeds, run counts, and environment notes.
\end{GovernanceBox}

\subsection{D5: Statistical / Analytical Plan}
\textbf{Required:} statistical tests/models, uncertainty quantification, power considerations (where applicable), correction procedures, calibration/validation approach.

\subsection{D6: Implementation \& System Design}
\begin{GovernanceBox}{Implementation Only (No New Math)}
D6 is the executable translation of M4/M6.
\textbf{Hard rule:} no new math definitions in D6; implementation must reference ALG-IDs and DOC01 symbol types.
\end{GovernanceBox}

\subsection{D7: Regulatory / Safety / Audit Layer (Cross-Cutting)}
\label{sec:d7_crosscutting}
\begin{GovernanceBox}{Cross-Cutting Audit Gates}
D7 is \textbf{not} a final chapter; it is an audit layer applied across blocks.
At minimum, define gates:
\begin{itemize}
  \item Privacy Gate (applies to D1, D4, DOC07),
  \item Ethics Gate (applies to D4),
  \item Security Gate (applies to D6, DOC07),
  \item Traceability Gate (applies to DOC04, DOC07, DOC05).
\end{itemize}
No RELEASE is valid without PASS for all applicable gates.
\end{GovernanceBox}

\begin{table}[h]
\centering
\begin{tabularx}{\textwidth}{@{}lX X X@{}}
\toprule
\textbf{Gate} & \textbf{Applies to} & \textbf{Evidence required} & \textbf{Verdict} \\
\midrule
D7-P (Privacy) & D1, D4, DOC07 & data minimization plan; legal basis; retention; anonymization/pseudonymization & PASS/FAIL/WAIVER \\
D7-E (Ethics) & D4 & consent/ethics review status; risk/benefit; inclusion/exclusion rationale & PASS/FAIL/WAIVER \\
D7-S (Security) & D6, DOC07 & threat model; access control; secret handling; vulnerability checklist & PASS/FAIL/WAIVER \\
D7-T (Traceability) & DOC04, DOC05, DOC07 & req-to-test matrix; artifact map; reproducibility manifest & PASS/FAIL/WAIVER \\
\bottomrule
\end{tabularx}
\caption{Minimum D7 audit gates and required evidence.}
\end{table}

% =============================================================================
\section{Audit Authority and Project-Specific Verdict Engines}
\label{sec:audit_authority}
This protocol allows project-specific instantiations of an \textbf{Audit Authority} that issues PASS/FAIL/WAIVER decisions for gates and releases. Such an instantiation may be manual (review board) or automated (lint+CI), but it must remain auditable through evidence manifests.

% =============================================================================
\section{Reproducibility and Evidence Manifests}
A compliant project must maintain:
\begin{itemize}
  \item an \textbf{artifact map} (DOC07),
  \item an \textbf{environment lock} (dependency versions; hardware notes for FULL),
  \item a \textbf{run ledger} (seeds, timestamps, dataset IDs, metrics),
  \item a \textbf{hash manifest} (SHA-256 of key artifacts).
\end{itemize}

\begin{TemplateBox}{Evidence Manifest (minimum fields)}
\begin{itemize}
  \item Protocol ID + version tag
  \item Timestamp (UTC) and local timezone
  \item Artifact list with SHA-256
  \item Seeds and run IDs for stochastic processes
  \item Acceptance thresholds used for verdicts (explicit)
\end{itemize}
\end{TemplateBox}

% =============================================================================
\section{External Alignment (Silent Citation)}
To ensure industry and academic compatibility, this protocol is aligned with:
\begin{itemize}
  \item \textbf{C4 Model} for architecture views (DOC07/D6),
  \item \textbf{ISO/IEC/IEEE 15288} for system lifecycle and assurance hooks (DOC07/D7),
  \item \textbf{IMRAD} and \textbf{Registered Reports} patterns for DOC05 and D4/D5 discipline,
  \item \textbf{Zachman-style multi-perspective coverage} as a consistency lens (optional check).
\end{itemize}
This is ``silent citation'': alignment is declared for transparency while the protocol remains self-contained.

% =============================================================================
\section{Acceptance Criteria for v1.0}
This protocol transitions to \textbf{v1.0 STABLE} only if:
\begin{enumerate}
  \item at least one \textbf{FULL-scale} project is completed under this protocol,
  \item all applicable D7 gates are PASS with evidence,
  \item the reproducibility manifest is complete and independently rebuildable,
  \item reviewer feedback is incorporated without violating Global Invariants (Section~\ref{sec:global_invariants}).
\end{enumerate}

% =============================================================================
\appendix
\section{Annex A: Templates}
\subsection{D3 Mapping Table Template}
\begin{longtable}{@{}p{2.4cm}p{2.7cm}p{1.8cm}p{2.2cm}p{2.1cm}p{2.1cm}p{2.2cm}p{2.0cm}@{}}
\toprule
Math Symbol & Domain Quantity (Obs/Lat) & Units/Scale & Measurement Method & Noise Model & Uncertainty Params & Sensitivity / Error Propagation & Validation Metric \\
\midrule
\endfirsthead
\toprule
Math Symbol & Domain Quantity (Obs/Lat) & Units/Scale & Measurement Method & Noise Model & Uncertainty Params & Sensitivity / Error Propagation & Validation Metric \\
\midrule
\endhead
\bottomrule
\endfoot
% Rows (fill per project)
\textit{(e.g., $x$)} & \textit{(e.g., BP\_sys (Obs))} & \textit{mmHg} & \textit{cuff device} & \textit{additive bias+noise} & \textit{$\sigma=...$} & \textit{$\partial y/\partial x$ ...} & \textit{calibration $R^2$} \\
\end{longtable}

\subsection{M7 Conjecture Entry Template}
\begin{TemplateBox}{M7 Conjecture Entry (minimum fields)}
\begin{itemize}
  \item CJ-ID; Statement; Dependencies
  \item Testability (how to falsify)
  \item Blocking Status: CRITICAL / IMPORTANT / OPTIONAL
  \item Owner; Due/Expiry Date
  \item Current Evidence; Next Action
  \item Verdict: OPEN / VERIFIED / FALSIFIED / DEPRECATED
\end{itemize}
\end{TemplateBox}

\section{Annex B: Zachman-style Consistency Lens (Optional)}
\begin{table}[h]
\centering
\begin{tabularx}{\textwidth}{@{}lX X X X X X@{}}
\toprule
Layer & What (Data) & How (Function) & Where (Location) & Who (People) & When (Time) & Why (Goal) \\
\midrule
Math-7 & objects/spaces & operators/algorithms & abstract spaces & (n/a) & complexity/time scales & metatheory/novelty \\
Domain-7 & variables/datasets & protocols/analysis & deployment context & users/subjects & study timeline & use-cases/objectives \\
DOC-7 & artifacts/registries & build/run scripts & repo/infra map & roles/owners & lifecycle & value/trade-offs \\
\bottomrule
\end{tabularx}
\caption{Optional completeness lens: avoid unintentionally empty cells.}
\end{table}

\section{Annex Z: Manifest and Evidence Requirements}
This annex is implemented operationally via the accompanying pack files:
\texttt{ANNEX\_Z\_MANIFEST\_SHA256.txt}, \texttt{protocol\_registry.sqlite}, and CI/build skeletons.

\newpage
\section{References \& Citation Policy (Silent Citation)}
\textbf{Principle:} The protocol is \emph{industry-compatible} and \emph{academically citable} without making external standards part of its dependency tree.
We therefore use \textbf{silent citation}: the main text says ``aligned with'' and the normative references are recorded here.

\subsection{How to cite this protocol}
\begin{quote}
Sahand Canon Standards Committee. \emph{DOC\_PROTO\_7\_PLUS\_14\_MAINSTREAM v0.3\_Integrated\_RC2: Specification \& Release Pack}. 2026.
\end{quote}

\subsection{Normative external references}
\begin{thebibliography}{9}
\bibitem{c4} S. Brown. \emph{The C4 model for visualising software architecture}. (Alignment reference for D6/DOC07 views).
\bibitem{iso15288} ISO/IEC/IEEE. \emph{15288: Systems and software engineering --- System life cycle processes}. (Alignment reference for DOC07/D7 lifecycle and audit gates).
\bibitem{imrad} IMRAD structure. \emph{Introduction, Methods, Results, and Discussion}. (Alignment reference for DOC05 reporting structure).
\bibitem{registered} Registered Reports. (Alignment reference for D4/D5 pre-specified analysis and p-hacking prevention).
\bibitem{zachman} Zachman Framework. (Reference ontology for coverage linting across What/How/Where/Who/When/Why).
\end{thebibliography}

\textbf{Plagiarism safeguard:} Alignment does \emph{not} claim authorship or ownership of the above external standards; it declares compatibility only.

\end{document}